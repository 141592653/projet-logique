\documentclass{article}
\usepackage[utf8]{inputenc}
\usepackage[francais]{babel}
\usepackage[T1]{fontenc}
\usepackage{tikz}
\usepackage{graphicx}
\usepackage[left=3cm,right=3cm,top=3cm,bottom=3cm]{geometry}

\title{Rapport du projet de logique}
\author{Julien Rixte}

\begin{document}
\maketitle

\section{Introduction}
L'objectif de ce projet était de calculer l'antécédent par la fonction de hachage MD5 d'un digest quelconque. Cet antécédent devait également satisfaire certaines contraintes. 
Dans ce rapport, je décrirai dans un premier temps la démarche que j'ai suivie. Puis, j'expliquerai la manière dont mon code est structuré. Enfin, je proposerai des améliorations possibles pour l'implémentation que j'ai écrite.

\section{Démarche}
Comme suggéré dans le sujet du projet, j'ai commencé par écrire les fonctions de manipulation de formules logiques. La principale fonction était la conversion d'une formule quelconque en formule en forme normale conjonctive. Bien que cette partie ne présente pas de réelle difficulté, il était fondamental de la tester correctement car je me suis rendu compte par la suite qu'il était difficile de déboguer le code lors de l'inversion de hachage. \par

Mes tests étaient d'ailleurs insuffisants car en inversant WeakHash, je me suis rendu compte, après une longue phase de débogage que la fonction de conversion n'était pas correcte. \par

Le temps que j'ai perdu sur cet erreur n'aura pas été totalement inutile car je me suis forcé par la suite à tester toutes les fonctions que j'écrivais, en particulier au moment de l'inversion de MD5. Le fait d'avoir à écrire ces tests m'a d'ailleurs obligé à écrire des fonctions locales sur des variables quelconques (et non uniquement sur les variables pour lesquelles elles sont destinées), ce qui les rend bien plus modulaires.

\section{Structure du code}
\subsection{Organisation des variables}
Le premier problème qui se posait était la manière d'organiser les variables. Comme le bootstrap le suggéré, mes variables sont simplement représentées par un entier. 
\par
Tout d'abord, les variables sont organisées en groupes de 32 bits. Un groupe de 32 bits est représenté par l'entier n correspondant à la première variable du groupe. Ainsi, $n + i$ donne la $i^{ème}$ variable du groupe représenté par n. 
\par
Ensuite, ces groupes (à l'exception des variables de l'input et des quatre additions finales de MD5) sont organisés en steps, chaque step représentant les variables nécessaires à l'inversion de ce step.
\newline

\begin{tabular}{|c|c|c|c|c|c|c|c|}
   \hline
   Input (16 blocs) & $a_0$,$c_0$,$d_0$ & Dernières additions (8 blocs)& $Step_0$ (6 blocs) & $Step_1$ & $Step_2$ & ... & $Step_k$   \\
   \hline
\end{tabular}
\newline
\par
Chaque step $s$ est composé de 6 blocs : 
\begin{itemize}
\item $b_s$ contient les variables correspondant aux calculs successifs de $b$ dans la boucle principale de md5. On peut aisément connaître $a_s$, $c_s$ et $d_s$ en remarquant que, en posant $b_{-1} = c_0$, $b_{-2} = d_0$ et $b_{-3} = a_0$ , on a $a_s = b_{s-3}$, $c_s = b_{s-1}$ et  $d_s = b_{s-3}$.
\item $non\_lin_s$ est le résultat de la fonction non linéaire
\item $carry41_s, carry42_s$ sont les deux retenues nécessaires à l'addition de quatre blocs.
\item $sum4$ est le résultat de l'addition des quatre blocs
\item $carry\_lr$ est la retenue pour l'opération d'addition-rotation. Il est à noter que l'on a pas besoin de variables représentant le résultat de l'addition-rotation car celui-ci sera stocké dans $b_{s+1}$
\end{itemize}

\vspace{0.4cm}
\begin{tabular}{|c|c|c|c|c|c|}
   \hline
    $b$& $non\_lin$ & $carry41$ & $carry42$ & $sum4$ & $carry\_lr$ \\
   \hline
\end{tabular}

\subsection{Découpage des formules}
J'ai découpé la formule permettant de casser en trois parties : 
\begin{itemize}
\item initialisation
\item steps
\item additions finales
\end{itemize}
 
L'initialisation consistait à l'origine à forcer la valeur de certaines variables, notamment les variables correspondant au digest, les quatre blocs $a_0$, $b_0$, $c_0$ et $d_0$ ainsi que les informations partielles sur l'input. Cependant, pour accélérer le calcul de minisat, le mieux est d'utiliser la substitiution afin de diminuer le nombre de clauses et de variables. Je me suis donc contenter d'initialiser l'information partielle sur l'input car la substitution ne suffit pas : on doit dire à minisat que certaines variables de l'input sont déjà connues.
\par
Chaque step $s$ est la conjonction des formules suivantes : 
\begin{itemize}
  \item Affectation : la formule d'affectation permet de rendre deux blocs de variables égaux. Elle permet de reproduire les trois affectation $d:= c, c:=b , a:= copy_d$ de la fonction MD5.
  \item Fonction non linéaire : cette formule permet d'inverser la fonction non linéraire correspondant au step qu'on est en train d'inverser. Le résultat est stické dans $non\_lin_s$.
  \item Addition de 4 blocs : permet d'additionner $a_s$, le résultat de la fonction non linéraire, la composante $s$ du vecteur $k$ et le bloc de l'input correspondant au step quel'on souhaite inverser. On aurait pu ici combiner simplement des additions de blocs deux à deux mais le nombre de variables aurait alors significativement augmenté : on aurait eu besoin de trois additions et donc de trois blocs pour les retenues et trois blocs pour les résultats. Ici, on divise donc par deux le nombre de variables utilisées en se contentant de deux blocs de retenue et d'un bloc de résultat. Le résultat de l'addition est stocké dans $sum4_s$.
  \item Addition-rotation : permet de combiner la rotation du résultat de l'addition de 4 blocs et l'addition de cette rotation à $b_s$. Le résultat est stocké dans $b_{s+1}$. Ici, on évite d'utiliser des variables intermédiaires pour stocker le résultat de la rotation car on peut directement faire la rotation au moment de l'addition. On remarque que cette fonction est également utilisée au moment de l'addition finale avec une valeur de rotation nulle.
\end{itemize}

\subsection{Implémentation des formules}
Chacune des formules énumérées ci-dessus est calculée par une fonction indépendante qui prend en paramètre les blocs considérés. L'affectation et la fonction non linéaire ne posent pas de difficultés : l'affectation est une conjonction de littéraux et la fonction non linéaire est déjà écrite sous forme logique dans MD5. Détaillons plutôt le calcul de la formule de l'addition de 4 blocs. 
\par
Pour chaque variable du bloc de résultat, j'ai besoin d'une conjonction de 32 formules. En effet, selon la valeur des trois termes additionnés et des deux retenues, la sommes ou les retenues seront différentes. Ainsi, pour chacune des distributions possibles pour les 5 variables sus-citées, on donne sous la forme d'une implication la valeur de vérité des retenues suivantes et de la somme.
\section{Bilan et améliorations}
Comparaison entre 3 addition de 2 blocs et une addition de 4 blocs. Est-ce que enlever a,c et d est rentable? Doit-on implémenter les autres fonctions non linéaires?

\section{Conclusion}




\end{document}